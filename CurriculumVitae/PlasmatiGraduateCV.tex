%%%%%%%%%%%%%%%%%%%%%%%%%%%%%%%%%%%%%%%%%
% Plasmati Graduate CV
% LaTeX Template
% Version 1.0 (24/3/13)
%
% This template has been downloaded from:
% http://www.LaTeXTemplates.com
%
% Original author:
% Alessandro Plasmati (alessandro.plasmati@gmail.com)
%
% License:
% CC BY-NC-SA 3.0 (http://creativecommons.org/licenses/by-nc-sa/3.0/)
%
% Important note:
% This template needs to be compiled with XeLaTeX.
% The main document font is called Fontin and can be downloaded for free
% from here: http://www.exljbris.com/fontin.html
%
%%%%%%%%%%%%%%%%%%%%%%%%%%%%%%%%%%%%%%%%%

%----------------------------------------------------------------------------------------
%	PACKAGES AND OTHER DOCUMENT CONFIGURATIONS
%----------------------------------------------------------------------------------------

\documentclass[a4paper,10pt]{article} % Default font size and paper size

\usepackage{fontspec} % For loading fonts
\defaultfontfeatures{Mapping=tex-text}
\setmainfont[SmallCapsFont = Fontin SmallCaps]{Fontin} % Main document font

\usepackage{xunicode,xltxtra,url,parskip} % Formatting packages

\usepackage[usenames,dvipsnames]{xcolor} % Required for specifying custom colors

\usepackage[big]{layaureo} % Margin formatting of the A4 page, an alternative to layaureo can be \usepackage{fullpage}
% To reduce the height of the top margin uncomment: \addtolength{\voffset}{-1.3cm}

\usepackage{hyperref} % Required for adding links	and customizing them
\definecolor{linkcolour}{rgb}{0,0.2,0.6} % Link color
\hypersetup{colorlinks,breaklinks,urlcolor=linkcolour,linkcolor=linkcolour} % Set link colors throughout the document

\usepackage{titlesec} % Used to customize the \section command
\titleformat{\section}{\Large\scshape\raggedright}{}{0em}{}[\titlerule] % Text formatting of sections
\titlespacing{\section}{0pt}{3pt}{3pt} % Spacing around sections

\usepackage[spanish]{babel} 
\usepackage[latin1]{inputenc} 

\begin{document}

\pagestyle{empty} % Removes page numbering

\font\fb=''[cmr10]'' % Change the font of the \LaTeX command under the skills section

%----------------------------------------------------------------------------------------
%	NAME AND CONTACT INFORMATION
%----------------------------------------------------------------------------------------

\par{\centering{\Huge Juan \textsc{Hernández García}}\bigskip\par} % Your name

\section{Datos Personales}

\begin{tabular}{rl}
\textsc{Lugar y Fecha de Nacimiento:} & Lorca(Murcia)  | 13 Julio 1990 \\
\textsc{Address:} & C/La Seda Edif/La Seda N7 2E, Lorca (Murcia) \\
\textsc{Phone:} & 617117423\\
\textsc{email:} & \href{mailto:juanhgw@gmail.com}{juanhgw@gmail.com}
\end{tabular}

%----------------------------------------------------------------------------------------
%	WORK EXPERIENCE 
%----------------------------------------------------------------------------------------

\section{Experiencia Laboral}

\begin{tabular}{r|p{11cm}}
\textsc{1 Año} & Colaborador de \textsc{Tertius Informática},
Granada \\
 & \emph{Desarrollador Software en proyecto Loook}\\
& \footnotesize{Apoyo en el análisis, codificación, diseño y depuración de la
aplicación de geolocalización para dispositivos móviles.}\\
\multicolumn{2}{c}{} \\

%------------------------------------------------


\textsc{8 Meses} & Investigador y Desarrollador Software de Interfaces
Cerebro-Máquina, \textsc{Departamento de Arquitectura y
Tecnología de Computadores}, Universidad de Granada. \\
& \emph{Investigación y desarrollo aplicaciones que hacen uso de parámetros
biométricos del usuario como las ondas cerebrales, o detección de
micromovimientos musculares.}
\\
& \emph{Dispositivos analizados: Neurobit Optima 4, MindWave Mobile,
GazeTracker.} \\
& \emph{Areas de investigación principales: EEG, EMG, EOG, EHG.}\\
& \footnotesize{}\\
\multicolumn{2}{c}{} \\

%------------------------------------------------

\textsc{6 Meses} & Diseñador y Desarrollador de Modelos de Simulación de
Fenómenos Físicos, \textsc{Departamento de Física Aplicada}, Universidad de
Granada. \\
& \emph{Desarrollador y diseñador de modelos de simulación realistas
para la facilitación del aprendizaje universitario. Se desarrollan múltiples
aplicaciones que hacen uso de la tecnología Applet de Java.}
\end{tabular}

%----------------------------------------------------------------------------------------
%	EDUCATION
%----------------------------------------------------------------------------------------

\section{Educación}

\begin{tabular}{rl}	
\textsc{Julio} 2014 & Título en Ingeniería Informática Superior, Universidad de
Granada.\\
& Proyecto Final de Carrera | Traductor y Lenguaje Específico para
Prototipado y \\
& Autogeneración Musical. \\
& Núm Menciones de Honor | 10. \\
&\\


%------------------------------------------------

\textsc{Mayo} 2013 & Desarrollo de Aplicaciones Móviles con Dispositivos ANDROID \\
& Horas | 120 \\
&\\

%------------------------------------------------

\textsc{Febrero} 2012 & Supuestos y Casos Prácticos de Configuración de Routers Y Switches CISCO \\
& Horas | 40 \\
&\\

%------------------------------------------------


\textsc{Marzo} 2010 & Curso WEB 2.0: Arquitectura Orientada a Servicios
en JAVA \\
& Horas | 50 \\
&\\

\end{tabular}


%----------------------------------------------------------------------------------------
%	COMPUTER SKILLS 
%----------------------------------------------------------------------------------------

\section{Habilidades Informáticas}

\begin{tabular}{rl}
Lenguajes de Programación: & \textsc{C\#}, \textsc{C++}, \textsc{C},
\textsc{php}, \textsc{html}, \textsc{Java}, \textsc{Pascal}, \textsc{Prolog},
\textsc{Ruby},  {\fb \LaTeX}\setmainfont[SmallCapsFont=Fontin
SmallCaps]{Fontin-Regular}\ldots \\


Entornos de Desarrollo: & \textsc{VisualStudio}, \textsc{NetBeans},
\textsc{Eclipse}, \textsc{DreamWeaver}, \textsc{TexnicCenter}\ldots. \\

Tecnologías Dominadas: & .Net, ANDROID, Java, WEB, Applet\ldots 

\\
\end{tabular}

%----------------------------------------------------------------------------------------
%	LANGUAGES
%----------------------------------------------------------------------------------------

\section{Idiomas}

\begin{tabular}{rl}
\textsc{Inglés:} & Nivel B2\\

\textsc{Francés:} & Nivel Básico\\

\textsc{Español:} & Lengua Materna\\
\end{tabular}


%----------------------------------------------------------------------------------------
%	INTERESTS AND ACTIVITIES
%----------------------------------------------------------------------------------------

\section{Intereses y Otras Actividades}


Programación, tecnologías emergentes, desarrollo de videojuegos,
biomedicina\\
integrante de grupo musical, balonmano (jugador), diseño web.

%----------------------------------------------------------------------------------------

\end{document}
